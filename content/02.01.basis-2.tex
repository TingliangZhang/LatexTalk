% !TeX encoding = UTF-8
% !TeX program = latexmk
% !TeX root = ../latex-talk.tex

\begin{frame}[fragile]{层次与目录生成}
    \begin{columns}
    \begin{column}{.6\textwidth}
    
    \begin{lstlisting}[basicstyle=\ttfamily\small]
    \tableofcontents % 这里是目录
    \part{有监督学习}
    \chapter{支持向量机}
    \section{支持向量机简介}
    \subsection{支持向量机的历史}
    \subsubsection{支持向量机的诞生}
    \paragraph{一些趣闻}
    \subparagraph{第一个趣闻}
    \end{lstlisting}
    \end{column}
    \begin{column}{.4\textwidth}
    第一部分\quad 有监督学习\\
    第一章\quad 支持向量机 \\
    1. 支持向量机简介 \\
    1.1 支持向量机的历史 \\
    1.1.1 支持向量机的诞生 \\
    一些趣闻  \\
    第一个趣闻
    \end{column}
    \end{columns}
    
    \end{frame}
    
    
    \begin{frame}[fragile]{列表与枚举}
    \begin{columns}
    \begin{column}{.6\textwidth}
    \begin{lstlisting}[basicstyle=\ttfamily\small]
    \begin{enumerate}
    \item \LaTeX{} 好处都有啥
      \begin{description}
        \item[好用] 体验好才是真的好
        \item[好看] 强迫症的福音
        \item[开源] 众人拾柴火焰高
      \end{description}
    \item 还有呢?
      \begin{itemize}
        \item 好处 1
        \item 好处 2
      \end{itemize}
    \end{enumerate}
    \end{lstlisting}
    \end{column}
    \begin{column}{.4\textwidth}
    {\small
    \begin{enumerate}
    \item \LaTeX{} 好处都有啥
      \begin{description}
        \item[好用] 体验好才是真的好
        \item[好看] 治疗强迫症
        \item[开源] 众人拾柴火焰高
      \end{description}
    \item 还有呢?
      \begin{itemize}
        \item 好处 1
        \item 好处 2
      \end{itemize}
    \end{enumerate}
    }
    \end{column}
    \end{columns}
    
    \end{frame}
    
    
    \begin{frame}[fragile]{交叉引用与插入插图}
      \begin{itemize}
      \item 给对象命名:图片、表格、公式等\\
      |\label{name}|
    \item 引用对象\\
      |\ref{name}|
      \end{itemize}
    \bigskip
    
      \begin{minipage}{0.7\linewidth}
        \begin{lstlisting}
    图书馆馆徽请参见图~\ref{fig:lib}。
    \begin{figure}[htbp]
      \centering
      \includegraphics[height=.2\textheight]%
      {libicon.pdf}
      \caption{图书馆馆徽。}
      \label{fig:lib}
    \end{figure}
    \end{lstlisting}
      \end{minipage}\hfill
      \begin{minipage}{0.3\linewidth}\centering
        {\songti 未来交互Logo请参见图~1。}\\[1em]
     \includegraphics[height=0.2\textheight]{FutureInteraction.pdf}\\
     {\footnotesize\heiti 图~1. 未来交互Logo。}
      \end{minipage}
    \end{frame}
    
    \begin{frame}[fragile]{交叉引用与插入表格}
      \begin{columns}
      \column{.6\textwidth}
      \begin{lstlisting}
    \begin{table}[htbp]
       \caption{编号与含义}
       \label{tab:number}
       \centering
       \begin{tabular}{cl}
         \toprule
         编号 & 含义 \\
         \midrule
         1    & 第一 \\
         2    & 第二 \\
         \bottomrule
       \end{tabular}
    \end{table}
    公式~(\ref{eq:vsphere}) 中编号与含义
    请参见表~\ref{tab:number}。
    \end{lstlisting}
    \column{.4\textwidth}
    \centering
    {\small 表~1. 编号与含义}\\[2pt]
    \begin{tabular}{cl}\toprule
    编号 & 含义 \\\midrule
    1 & 第一\\
    2  & 第二\\\bottomrule
    \end{tabular}\\[5pt]
    
    \normalsize 公式~(\ref{eq:vsphere})编号与含义请参见表~1。
      \end{columns}
    \end{frame}
    
    \begin{frame}[fragile]{浮动体}
    \begin{itemize}
    \item 初学者最“捉摸不透”的特性之一 \link{https://liam.page/2017/03/11/floats-in-LaTeX-basic}
    \item 图片和表格有时会很大,在插入的位置不一定放得下,因此需要浮动调整
    \item 避免在文中使用「下图」「上图」的说法,而是使用图表的编号,例如 |图~\ref{fig:fig1}| 。
    \item |\begin{figure}[<位置>] 图片 \end{figure}|
      \begin{itemize}
      \item 位置参数指定浮动体摆放的偏好
      \item |h| 当前位置(here), |t| 顶部(top), |b| 底部(bottom), |p| 单独成页(p)
      \item |!h| 表示忽略一些限制,|H| 表示强制\alert{(强烈不建议,除非你知道自己在做什么)}
      \end{itemize}
    \item 温馨提示:图标题一般在下方,表标题一般在上方
    \end{itemize}
    \end{frame}
    
    \begin{frame}[fragile]
      \frametitle{作图与插图}
      \begin{itemize}
        \item 外部插入
    
          \begin{itemize}
            \item Mathematica、MATLAB
            \item PowerPoint、Visio、Adobe Illustrator、Inkscape
            \item Python \pkg{Matplotlib} 库、\texttt{Plots.jl}、R、Plotly 等
            \item draw.io \link{https://draw.io/}、ProcessOn \link{https://www.processon.com/} 等在线绘图网站
          \end{itemize}
    
        \item \TeX{} 内联
    
          \begin{itemize}
            \item Asymptote
            \item \alert{\pkg{pgf}/\pkg{TikZ}、\pkg{pgfplots}}
          \end{itemize}
    
        \item 插图格式
    
          \begin{itemize}
            \item 矢量图:|.pdf|
            \item 位图:|.jpg| 或 |.png|
            \item \alert{不再推荐 \texttt{.eps}}
            \item 不(完全)支持 |.svg|、|.bmp|
          \end{itemize}
    
        \item 一些参考:\link{https://www.zhihu.com/question/21664179}
                        \link{https://tex.stackexchange.com/q/158668}
                        \link{https://tex.stackexchange.com/q/72930}
      \end{itemize}
    \end{frame}
    
    \begin{frame}[fragile]{表格绘制}
      \begin{itemize}
        \item 使用 \pkg{booktabs}、\pkg{longtables}、\pkg{multirow} 等宏包
        \item 手动绘制表格确实比较令人头疼,且较难维护
        \item 有编程基础可以使用 Python、R 等语言生成表格代码
        \item 推荐使用在线工具绘制后导出代码:\LaTeX{} Table Generator \link{https://www.tablesgenerator.com/latex_tables}
      \end{itemize}
    \end{frame}

    \begin{frame}[fragile]{引用和参考文献}
      \begin{columns}
        \begin{column}{0.45\textwidth}
          \begin{itemize}
            \item \LaTeX{} 中的参考文献由\emph{文献数据库}(即 |.bib| 文件)生成
            \item 向数据库添加文献条目的方法:
              \begin{itemize}
                \item 使用 Mendely、Zotero、NoteExpress 等软件导出为 |.bib| 文件
                \item 从 Google Scholar、MathSciNet、ACM DL 等在线数据库导出
                \item 手工编写条目(不推荐)
              \end{itemize}
          \end{itemize}
        \end{column}
        \begin{column}{0.55\textwidth}
          \begin{lstlisting}[basicstyle=\footnotesize\ttfamily]
    @article{mellinger1996laser,
      author     = {Mellinger, A and Vidal, C R and Jungen, {Ch}},
      title      = {Laser reduced fluorescence study of the carbon monoxide nd triplet Rydberg series},
      journal    = {J Chem Phys},
      year       = {1996},
      volume     = {104},
      pages      = {8913--8921},
    }
            \end{lstlisting}
        \end{column}
      \end{columns}
      \begin{itemize}
        \item 在正文中使用 |\cite{key1, key2}| 引用条目,并在最后使用 |\bibliography{bibfile}| 命令打印参考文献列表
        \item \BibTeX{} 可生成各种不同格式的引用和参考文献(需要宏包支持):APA、MLA、GB/T 7714 等
        \end{itemize}
    \end{frame}
    
    \begin{frame}[fragile]
      \frametitle{宏包推荐(\textbf{先读文档}后使用)}
      \setlength{\leftmarginii}{1.5em}
      \vspace{-1.5em}
      \begin{multicols}{3}
        \begin{itemize}
          \item 必备
    
            \begin{itemize}
              \item \pkg{amsmath}
              \item \pkg{graphicx}
              \item \pkg{hyperref}
            \end{itemize}
    
          \item 样式
    
            \begin{itemize}
              \item \pkg{caption}
              \item \pkg{enumitem}
              \item \pkg{fancyhdr}
              \item \pkg{footmisc}
              \item \pkg{geometry}
              \item \pkg{titlesec}
            \end{itemize}
    
          \item 数学
    
            \begin{itemize}
              \item \pkg{bm}
              \item \pkg{mathtools}
              \item \pkg{physics}
              \item \pkg{unicode-math}
            \end{itemize}
    
          \item 表格
    
            \begin{itemize}
              \item \pkg{array}
              \item \pkg{booktabs}
              \item \pkg{longtable}
              \item \pkg{tabularx}
            \end{itemize}
    
          \item 插图、绘图
    
            \begin{itemize}
              \item \pkg{float}
              \item \pkg{pdfpages}
              \item \pkg{standalone}
              \item \pkg{subfig}
              \item \pkg{pgf}/\pkg{tikz}
              \item \pkg{pgfplots}
            \end{itemize}
    
          \item 字体
    
            \begin{itemize}
              \item \pkg{newpx}
              \item \pkg{pifont}
              \item \pkg{fontspec}
            \end{itemize}
    
          \item 各种功能
    
            \begin{itemize}
              \item \pkg{algorithm2e}
              \item \pkg{beamer}
              \item \pkg{biblatex}
              \item \pkg{listings}
              \item \pkg{mhchem}
              \item \pkg{microtype}
              \item \pkg{minted}
              \item \pkg{natbib}
              \item \pkg{siunitx}
              \item \pkg{xcolor}
            \end{itemize}
    
          \item 多语言
    
            \begin{itemize}
              \item \pkg{babel}
              \item \pkg{polyglossia}
              \item \pkg{ctex}
              \item \pkg{xeCJK}
            \end{itemize}
        \end{itemize}
      \end{multicols}
      \vspace*{-0.5cm}
    \end{frame}

%%% vim: set sw=2 isk+=\: et tw=80 cc=+1 formatoptions+=mM: