% !TeX encoding = UTF-8
% !TeX program = latexmk
% !TeX root = ../latex-talk.tex

\begin{frame}[fragile]
  \frametitle{引擎与格式}
  \begin{itemize}
    \item \textbf{引擎}:\TeX{} 的实现

      \begin{itemize}
        \item \pdfTeX{}:直接生成 PDF,支持 micro-typography
        \item \XeTeX{}:支持 Unicode、OpenType 与复杂文字编排(CTL)
        \item \LuaTeX{}:支持 Unicode,内联 Lua,支持 OpenType
        \item (u)p\TeX{}:日本方面推动,生成 |.dvi|,(支持 Unicode)
        \item Ap\TeX{}:底层 CJK 支持,内联 Ruby,Color Emoji
      \end{itemize}

    \item \textbf{格式}:\TeX{} 的语言扩展(命令封装)

      \begin{itemize}
        \item plain \TeX{}:Knuth 最早的实现,底层、细节较多
        \item \LaTeX{}:排版科技类文章的事实标准,相对易用
        \item Con\TeX t:基于 \LuaTeX{} 实现,便于实现高级功能
      \end{itemize}

    \item \textbf{程序}:引擎 + dump 后的格式代码

      \begin{itemize}
        \item \alert{英文文章:\pdfLaTeX{}、\XeLaTeX{} 或 \LuaLaTeX{}}
        \item \alert{中文文章:\XeLaTeX{} 或 \LuaLaTeX{}}
      \end{itemize}
  \end{itemize}
\end{frame}

\begin{frame}[fragile]
  \frametitle{编译}
  \begin{itemize}
    \item 现代 \TeX{} 引擎均可直接生成 PDF
    \item 命令行

      \begin{itemize}
        \item |pdflatex|/|xelatex|/|lualatex| + |<文件名>[.tex]|
        \item 多次编译:每次均需要读取并处理中间文件
        \item 推荐 \pkg{latexmk}:运行 |latexmk [<选项>] <文件名>| 即可自动完成所有工作
      \end{itemize}

    \item 编辑器

      \begin{itemize}
        \item 按钮的背后仍然是命令
        \item |PATH| 环境变量:确定可执行文件的位置
        \item VS Code + \LaTeX{} Workshop:配置 |tools| 和 |recipes|
      \end{itemize}
  \end{itemize}
\end{frame}


\begin{frame}[fragile]{文件结构}
  \lstset{language=[LaTeX]TeX}
  \begin{lstlisting}[basicstyle=\ttfamily]
\documentclass[a4paper]{ctexart}
% 文档类型,如 ctexart,[]内是选项,如 a4paper
% 这里开始是导言区
\usepackage{graphicx} % 引用宏包
\graphicspath{{fig/}} % 设置图片目录
% 导言区到此为止
\begin{document}
这里开始是正文
\end{document}
  \end{lstlisting}
\end{frame}

\begin{frame}[fragile]{\LaTeX{}“命令”}
  \framesubtitle{\emph{宏} (Macro)、或者\emph{控制序列} (control sequence)}
\begin{itemize}
\item 简单命令
  \begin{itemize}
    \item \verb|\命令|\hspace{2em}
    \verb|{\songti 中国人民解放军}| ~$\Rightarrow$ {\songti 中国人民解放军}
  \item \verb|\命令[可选参数]{必选参数}|\\
\verb|\section[精简标题]{这个题目实在太长了放到目录里面不太好看}|\\
$\Rightarrow$ {\heiti 1.1 \hspace{1em} \songti 这个题目实在太长了放到目录里面不太好看}
  \end{itemize}
\item 环境
  \begin{columns}[c]
  \begin{column}{0.45\textwidth}
    \begin{lstlisting}[basicstyle=\ttfamily]
\begin{equation*}
  a^2-b^2=(a+b)(a-b)
\end{equation*}
\end{lstlisting}
\end{column}\hspace{1em}
  \begin{column}{0.45\textwidth}
$ a^2-b^2=(a+b)(a-b)$
\end{column}
  \end{columns}
\end{itemize}
\end{frame}

\begin{frame}[fragile]{\LaTeX{} 常用命令}
  \begin{exampleblock}{简单命令}
\centering
\footnotesize
  \begin{tabular}{llll}
    \cmd{chapter} & \cmd{section} & \cmd{subsection} & \cmd{paragraph} \\
    章 & 节 & 小节 & 带题头段落 \\\hline
    \cmd{centering} & \cmd{emph} & \cmd{verb} & \cmd{url} \\
   居中对齐         &  强调      & 原样输出   & 超链接 \\\hline
  \cmd{footnote} & \cmd{item} & \cmd{caption} & \cmd{includegraphics} \\
   脚注 & 列表条目 & 标题 & 插入图片 \\\hline
  \cmd{label} & \cmd{cite} & \cmd{ref} \\
  标号 & 引用参考文献 & 引用图表公式等\\\hline
  \end{tabular}
\end{exampleblock}
\end{frame}
\begin{frame}[fragile]{\LaTeX{} 常用命令}
\begin{exampleblock}{环境}
\centering
\footnotesize
\begin{tabular}{lll}
  \env{table} & \env{figure} & \env{equation}\\
  表格 & 图片 & 公式 \\\hline
  \env{itemize} & \env{enumerate} & \env{description}\\
  无编号列表 & 编号列表 & 描述 \\\hline
\end{tabular}
\end{exampleblock}
\end{frame}
%
\begin{frame}{\LaTeX{}命令举例}
\cmdxmp{chapter}{前言}{\heiti 第 1 章\hspace{1em} 前言}
\cmdxmp{section[精简标题]}{这个题目实在太长了放到目录里面不太好看}{\heiti 1.1
  \hspace{1em} 这个题目实在太长了放到目录里面不太好看}
\cmdxmp{footnote}{我是可爱的脚注}{前方高能\footnote{我是可爱的脚注}}
\end{frame}

\begin{frame}[fragile]{\LaTeX{} 环境举例}
  \begin{minipage}{0.4\linewidth}
    \begin{lstlisting}[basicstyle=\ttfamily\small]
\begin{itemize}
  \item 一条
  \item 次条
  \item 这一条可以分为 ...
    \begin{itemize}
      \item 子一条
    \end{itemize}
\end{itemize}
\end{lstlisting}
  \end{minipage}\hspace{1.5cm}
  \begin{minipage}{0.4\linewidth}
\begin{itemize}
  \item 一条
  \item 次条
  \item 这一条可以分为 ...
    \begin{itemize}
      \item 子一条
    \end{itemize}
\end{itemize}
  \end{minipage}
\medskip

  \begin{minipage}{0.4\linewidth}
\begin{lstlisting}
\begin{enumerate}
  \item 一条
  \item 次条
  \item 再条
\end{enumerate}
\end{lstlisting}
  \end{minipage}\hspace{1.5cm}
  \begin{minipage}{0.4\linewidth}
\begin{enumerate}
  \item 一条
  \item 次条
  \item 再条
\end{enumerate}
  \end{minipage}
\end{frame}
%

\begin{frame}[fragile]{\LaTeX{} 数学公式}

\begin{columns}
\begin{column}{.5\textwidth}
\begin{lstlisting}[basicstyle=\ttfamily\small]
$V = \frac{4}{3}\pi r^3$

\[
  V = \frac{4}{3}\pi r^3
\]

\begin{equation}
\label{eq:vsphere}
V = \frac{4}{3}\pi r^3
\end{equation}
\end{lstlisting}
\end{column}

\begin{column}{.5\textwidth}
$V = \frac{4}{3}\pi r^3$

\[
  V = \frac{4}{3}\pi r^3
\]

\begin{equation}
\label{eq:vsphere}
V = \frac{4}{3}\pi r^3
\end{equation}
\end{column}
\end{columns}

\end{frame}

\begin{frame}[fragile]{\LaTeX{} 数学公式}
\begin{itemize}
\item 数学公式排版是 \LaTeX{} 的绝对强项
\item 数学排版需要进入数学模式,引用 \texttt{amsmath} 宏包
	\begin{itemize}
	\item 用单个美元符号(\verb|$|) 包围起来的内容是 {\bf 行内公式}
  \item 用两个美元符号(\verb|$$|) (不推荐)或 \verb|\[ \]| 包围起来的是 {\bf 单行公式} 或 {\bf 行间公式}
	\item 使用数学环境,例如 \texttt{equation} 环境内的公式会自动加上编号,
		\texttt{align} 环境用于多行公式(例如方程组、多个并列条件等)
  \end{itemize}
\item 寻找符号
    \begin{itemize}
      \item 运行 \texttt{texdoc symbols} 查看符号表
      \item S. Pakin. \emph{The Comprehensive \LaTeX{} Symbol List}
            \link{https://ctan.org/pkg/comprehensive}
      \item 手写识别(有用但不全面):Detexify \link{http://detexify.kirelabs.org}
    \end{itemize}
\item MathType 也可以使用和导出 \LaTeX{} 公式(不推荐)
\end{itemize}
\end{frame}

\begin{frame}[fragile,label={frame:unicode-math}]{unicode-math:现代的数学输入方式}
\LaTeX{} 的公式确实很强大,但是……符号有点难记?

\pkg{unicode-math} 宏包提供了几乎所见即所得的公式输入(\ThuThesis 默认使用):

\begin{itemize}
  \item 可直接输入各类符号对应的 Unicode 字符(需要使用 UTF-8 编码):
  \begin{align*}
    𝐹(𝑠) &= ℒ\{𝑓(𝑡)\} = ∫_0^∞ e^{−𝑠𝑡} 𝑓(𝑡) d𝑡 \\
    𝐁 &= 𝜇_0(𝐌 + 𝐇)
  \end{align*}
  \item 使用 |symbf| 等命令自动处理字母的粗体、斜体等变体,不必引入额外宏包。
\end{itemize}

\begin{columns}[c]
  \begin{column}{0.45\textwidth}
    \begin{lstlisting}[basicstyle=\ttfamily]
\begin{align*}
\symbf{\beta} &= \beta \symbf{I} \\
\symbf{a} &= a \symbf{I}
\end{align*}
\end{lstlisting}
\end{column}\hspace{1em}
  \begin{column}{0.45\textwidth}
  \begin{align*}
      \symbf{\beta} &= \beta \symbf{I} \\
      \symbf{a} &= a \symbf{I}
  \end{align*}
  \end{column}
\end{columns}

\end{frame}




%%% vim: set ts=2 sts=2 sw=2 isk+=\: et tw=80 cc=+1 formatoptions+=mM:
