\documentclass[xcolor=table,dvipsnames,svgnames,aspectratio=169,fontset=fandol,punct=kaiming]{ctexbeamer}
% Author: alick<alick9188@gmail.com>
% Author: justin <justin.w.xd@gmail.com>
% Author: Harry Chen <harry-chen@outlook.com>

% This file is modified from a solution template for:

% - Giving a talk on some subject.
% - The talk is between 15min and 45min long.
% - Style is ornate.

% Copyright 2004 by Till Tantau <tantau@users.sourceforge.net>.
%
% In principle, this file can be redistributed and/or modified under
% the terms of the GNU Public License, version 2.
%
% However, this file is supposed to be a template to be modified
% for your own needs. For this reason, if you use this file as a
% template and not specifically distribute it as part of a another
% package/program, I grant the extra permission to freely copy and
% modify this file as you see fit and even to delete this copyright
% notice.

\usepackage{tikz}

\graphicspath{{fig/}}

\mode<presentation>
{
  \usetheme{material}
  \renewcommand{\MaterialIcon}{FutureInteraction.png}
  \usefonttheme[onlymath]{serif}
  \setbeamercovered{transparent=5}
  \setbeamercolor*{structure}{fg=primaryD}
  \setbeamertemplate{itemize item}{\raise1.25pt\hbox{\tikz\draw[fill=fg] (0,0) circle (.3ex);}}
  \setbeamertemplate{itemize subitem}{\color{fg}\tiny\raise1.25pt\hbox{\donotcoloroutermaths$\blacktriangleright$}}
  \setbeamertemplate{itemize subsubitem}{\raise2.5pt\hbox{\tikz\draw[fill=fg] (0,0) rectangle (.7ex, .2ex);}}

  \setlength\leftmargini{1.4em}
  \setlength\leftmarginii{1.4em}
  \setlength\leftmarginiii{1.4em}
  \setbeamersize{description width=0.24cm}
}

\usepackage{mflogo} % for \MF, \MP
\usepackage{graphicx}
\usepackage{listingsutf8}
\usepackage{xspace}
\usepackage{amsmath}
\usepackage{calligra}
\usepackage{fontspec}
\usepackage{ccicons}
\usepackage{hologo}
\usepackage{colortbl}
\usepackage{pstricks}
\usepackage{pst-node}
\usepackage{hyperxmp}
\usepackage{booktabs}
\usepackage{qrcode}
\usepackage{tipa}
\usepackage{multicol}
\usepackage[normalem]{ulem}
\usepackage{stackengine}
\usepackage{fontawesome}

\hypersetup{
  pdfauthor={Tina Zhang, Harry Chen, Justin Wong, Alick Zhao},
  pdfcopyright={Copyright (C) 2015-2019 by Alick Zhao, Justin Wong, Harry Chen and Tina Zhang.
  Licensed under CC-BY-SA 4.0. Some rights reserved.},
  pdflicenseurl={http://creativecommons.org/licenses/by-sa/4.0/},
}

% From thuthesis user guide
\makeatletter
\def\psRotation#1(#2,#3)#4{%
  \rput{#1}(#2,#3){%
    \psellipticarc[linewidth=.4pt]{->}(0,-0.1)(0.6,0.15){120}{70}
    \ifdim#1pt>\z@\rput[l]{*0}(0.675,0){#4}\else\rput[l](0.675,0){#4}\fi
  }%
}
\makeatother

% For tipa to work.
\newfontfamily\useTIPAfont{Times New Roman}

% xeCJK conf setup
\renewcommand\CJKfamilydefault{\CJKsfdefault} % for slides

\setCJKsansfont[Path=Fonts/]{wqy-microhei.ttc}
\setCJKmonofont[Path=Fonts/]{wqy-microhei.ttc}

%\setCJKsansfont{WenQuanYi Micro Hei}
%\setCJKmonofont{WenQuanYi Micro Hei Mono}

\renewcommand{\TeX}{\hologo{TeX}}
\renewcommand{\LaTeX}{\hologo{LaTeX}}
\newcommand{\BibTeX}{\hologo{BibTeX}}
\newcommand{\XeTeX}{\hologo{XeTeX}}
\newcommand{\pdfTeX}{\hologo{pdfTeX}}
\newcommand{\LuaTeX}{\hologo{LuaTeX}}
\renewcommand{\CTeX}{C\TeX}
\newcommand{\MiKTeX}{\hologo{MiKTeX}}
\newcommand{\MacTeX}{Mac\hologo{TeX}}
\newcommand{\beamer}{\textsc{beamer}}
\newcommand{\XeLaTeX}{\hologo{Xe}\kern-.13em\LaTeX{}}
\newcommand{\pdfLaTeX}{pdf\LaTeX{}}
\newcommand{\LuaLaTeX}{Lua\LaTeX{}}

\def\TeXLive{\TeX{} Live\xspace}
\let\TL=\TeXLive
\newcommand{\ThuThesis}{\textsc{ThuThesis}\xspace}

\newcommand\link[1]{\href{#1}{\faLink}}
\newcommand\pkg[1]{\texttt{#1}}

% From thuthesis user guide.
\def\cmd#1{\texttt{\color{DarkBlue}\footnotesize $\backslash$#1}}
\def\env#1{\texttt{\color{DarkBlue}\footnotesize #1}}
\def\cmdxmp#1#2#3{\small{\texttt{\color{DarkBlue}$\backslash$#1}\{#2\}\hspace{1em}\\ $\Rightarrow$\hspace{1em} {#3}\par\vskip1em}}


\lstset{
  language=[LaTeX]TeX,
	basicstyle=\ttfamily\footnotesize,
	tabsize=2,
  keywordstyle=\bfseries\ttfamily\color{primaryD},
	commentstyle=\sl\ttfamily\color[RGB]{100,100,100},
	stringstyle=\ttfamily\color[RGB]{50,50,50},
	extendedchars=true,
	breaklines=true,
}
\lstdefinestyle{style@inline}{
  basicstyle   = \ttfamily,
  keepspaces   = true
}
\lstMakeShortInline[style=style@inline]|

\title
{如何使用 \LaTeX 排版论文}

\author[张庭梁] % (optional, use only with lots of authors)
{张庭梁\\ \texttt{zhangtl16@mails.tsinghua.edu.cn}}

\institute{清华大学未来通信兴趣团队}
% - Use the \inst command only if there are several affiliations.
% - Keep it simple, no one is interested in your street address.

\date[交互十讲]{2019 年 12 月}

\subject{LaTeX, paper, ThuThesis}

% Delete this, if you do not want the table of contents to pop up at
% the beginning of each subsection:
\AtBeginSubsection[]
{
  \begin{frame}<beamer>{目录}
    \tableofcontents[currentsection,currentsubsection]
  \end{frame}
}


% If you wish to uncover everything in a step-wise fashion, uncomment
% the following command:

%\beamerdefaultoverlayspecification{<+->}

%\hypersetup{
%pdfpagemode=FullScreen,
%}

\logo{\includegraphics[height=.15\textheight]{Tsinghua.png}}

\begin{document}

\begin{frame}
  \titlepage
\end{frame}

\begin{frame}{目录}
  \tableofcontents
  % You might wish to add the option [pausesections]
\end{frame}


% Since this a solution template for a generic talk, very little can
% be said about how it should be structured. However, the talk length
% of between 15min and 45min and the theme suggest that you stick to
% the following rules:

% - Exactly two or three sections (other than the summary).
% - At *most* three subsections per section.
% - Talk about 30s to 2min per frame. So there should be between about
%   15 and 30 frames, all told.

\include{introduction}
\include{basis}
\include{thuthesis}
\section{总结}

\begin{frame}{常见问题}
  \begin{itemize}
    \item \alert{编译不通过} 缺少必要宏包,命令拼写错误,括号未配对等
    \item \alert{表格图片乱跑} \LaTeX{} 自身的浮动定位算法
    \item \alert{段落间距变大} \LaTeX{} 排版算法
    \item \alert{参考文献} 推荐使用 \BibTeX{} 或者 Bib\LaTeX{},也可以手写 \cmd{bibitem} \link{https://github.com/hushidong/biblatex-gb7714-2015}
  \end{itemize}
\end{frame}

\begin{frame}{系统学习}
  \begin{itemize}
      \item 包太雷 《\LaTeX{} Notes(第二版)》~(3小时)(lnotes2) \link{http://dralpha.altervista.org/zh/tech/lnotes2.pdf}
      \item Stefan Kottwitz 《LaTeX Cookbook》
      \item WikiBooks:英文 \link{https://en.wikibooks.org/wiki/LaTeX}、中文 \link{https://zh.wikibooks.org/wiki/LaTeX}
      \item 在线教程:ShareLaTeX、OverLeaf 都有帮助
      \item 经典文档
        \begin{itemize}
          \item 仔细阅读《一份不太简短的~\LaTeXe{} 介绍》(lshort-zh)~(1--2 天)
          \item 粗略阅读《\LaTeXe{} 插图指南》~(2--3 小时)
        \end{itemize}
      \item 仔细阅读《\ThuThesis{} 用户手册》~(20 分钟)
      \item 从~\ThuThesis{} 示例文档入手
  \end{itemize}
\end{frame}

\begin{frame}{扩展阅读}
  \begin{itemize}
    \item 一份其实很短的 \LaTeX 入门文档 (Liam Huang) \link{https://liam.page/2014/09/08/latex-introduction/}
    \item 网站推荐:
      \begin{itemize}
        \item http://www.latexstudio.net/
        \item http://www.chinatex.org/
      \end{itemize}
    \item 知乎 LaTeX 专栏: http://zhuanlan.zhihu.com/LaTeX
    \item \ThuThesis{}使用示例文档(模板自带)
    \item \LaTeX{}杂谈(刘海洋)
    \item 《\LaTeX{}入门》(刘海洋)
    \item 现代 LaTeX 入门讲座(曾祥东)\link{https://github.com/stone-zeng/latex-talk/releases/tag/2019-04-18}
  \end{itemize}
\end{frame}


\begin{frame}{利用文档}
  \begin{itemize}
    \item 常用文档
      \begin{itemize}
        \item symbols: 符号大全
        \item Mathmode: 数学参考
        \item ctex, xeCJK: 中文支持
        \item texlive-zh: \TL 安装与使用
        \item 所用宏包文档
      \end{itemize}
    \item 工具
      \begin{itemize}
        \item tlmgr: \TL 管理器
        \item texdoc: \TeX{} 文档查看器\\
          例如:\texttt{texdoc lshort-zh}
        \item 在线文档 \TeX{}doc \link{http://texdoc.net/}
        \item TeX Studio 和 WinEdt 都支持在帮助里看文档
      \end{itemize}
  \end{itemize}
\end{frame}

\begin{frame}{一点人生的经验}
  \begin{itemize}
    \item 不要着急安装,先在 OverLeaf 上熟悉各类操作
    \item 不要过于相信网上的中文文档
      \begin{itemize}
        \item 简单鉴别方法: 排版的好看程度
      \end{itemize}
    \item 湿兄用U盘拷给你的的 C\TeX 套装一定是过时的,ThuThesis 八成是老版本的
    \item 如果你要处理中文
      \begin{itemize}
        \item 使用 XeLaTeX, 使用 XeLaTeX, 使用 XeLaTeX
        \item 忘记 CJK, 忘记 CJK, 忘记 CJK
        \item 使用 ctex 宏包(2.0以上版本)(跟 C\TeX 套装仅仅是名字像)
      \end{itemize}
    \item 写一点,编译一次,减小排错搜索空间
  \end{itemize}
\end{frame}

\begin{frame}[fragile]
  \frametitle{Git版本管理}
  \begin{itemize}
    \item 版本管理的必要性
      \begin{itemize}
        \item 远离「初稿,第二稿……终稿,终稿(打死也不改了)」命名
        \item 方便与他人协同合作
      \end{itemize}
    \item 基本用法
      \begin{itemize}
        \item 跟踪更改:|git init|、|git add|、|git commit|
        \item 撤销与回滚:|git reset|、|git revert|
        \item 分支与高级用法:|git branch|、|git checkout|、|git rebase|
        \item 远端仓库操作:|git pull|、|git push|、|git fetch|
        \item 推荐用 VS Code 等进行可视化操作
        \item 参考链接:\link{https://git-scm.com/book/en/v2}
          \link{https://www.liaoxuefeng.com/wiki/0013739516305929606dd18361248578c67b8067c8c017b000}
      \end{itemize}
    \item 在线 Git 服务
      \begin{itemize}
        \item GitHub \href{https://github.com}{\faGithub}
        \item 清华大学代码托管服务(基于 GitLab) \link{https://git.tsinghua.edu.cn}
      \end{itemize}
  \end{itemize}
  \end{frame}

% 寻求帮助
\begin{frame}{求助}
  \begin{columns}[c]
    \begin{column}{.45\textwidth}
      \begin{itemize}
        \item BBS
          \begin{itemize}
            \item 水木社区 TeX 版 \link{http://www.newsmth.net/nForum/board/TeX}
            \item CTEX 社区\link{http://bbs.ctex.org/}(从2018年底开始无限期关闭)
          \end{itemize}
        \item UK FAQ \link{http://www.tex.ac.uk/cgi-bin/texfaq2html}
        \item TeX StackExchange \link{https://tex.stackexchange.com/}
        \item Google, Bing, etc.
          \begin{itemize}
            \item 使用\textbf{英语}搜索
          \end{itemize}
      \end{itemize}
    \end{column}
    \begin{column}{.45\textwidth}
      \includegraphics[width=\textwidth]{TFZsuperellipse-crop.pdf}
    \end{column}
  \end{columns}
\end{frame}

\begin{frame}{你也可以帮助}
  \begin{itemize}
    \item 错误反馈、改进建议:GitHub Issues
    \item 出力维护:LaTeX 宏包编写、Git
    \item 科普、答疑 \hspace{2em}\sout{来当主讲人}
  \end{itemize}
\end{frame}

%%% vim: set ts=2 sts=2 sw=2 isk+=\: et tw=80 cc=+1 formatoptions+=mM:


\section*{附录}


\begin{frame}
  \begin{itemize}
    \item 本幻灯片源码:
      \begin{itemize}
        \item \url{https://github.com/TingliangZhang/LatexTalk}
      \end{itemize}
    \item 本幻灯片参考:
      \begin{itemize}
        \item \url{https://github.com/tuna/thulib-latex-talk}
        \item \url{http://github.com/alick/fad-texlive-talk}
        \item \url{https://github.com/stone-zeng/latex-talk}
        \item \ThuThesis{}使用向导 v3.0
      \end{itemize}
    \item 许可证:CC BY-SA 4.0 Unported \ccbysa
    \item 幻灯片下载 \link{https://github.com/TingliangZhang/LatexTalk/blob/master/latex-talk.pdf} \hspace{1em}
      \qrcode[hyperlink, height=2cm]{https://github.com/TingliangZhang/LatexTalk/blob/master/latex-talk.pdf}
  \end{itemize}
\end{frame}

\begin{frame}
  \begin{center}
    {\Huge\calligra Thank you!}
  \end{center}
\end{frame}

\end{document}

%%% vim: set ts=2 sts=2 sw=2 isk+=\: et tw=80 cc=+1 formatoptions+=mM:
